\documentclass{article}
\usepackage[left=35mm,margin=20mm]{geometry}
\usepackage{ifthen,empheq}
\date{\today}
\usepackage{fancyhdr}
\usepackage{tikz}
\usetikzlibrary{calc,graphs,arrows}
\setlength{\headheight}{15pt}

\lhead{Assignment 1}
\chead{Dilawar Singh}
\rhead{\today}

\lfoot{}\cfoot{\thepage}\rfoot{}
\pagestyle{fancy}

\ifx\pdfoutput\undefined                         %LaTeX
  \RequirePackage[ps2pdf,bookmarks=true]{hyperref}
  \hypersetup{ %
    pdfauthor   = {\@author},
    pdftitle    = {\@title},
    pdfcreator  = {LaTeX with hyperref package},
    pdfproducer = {dvips + ps2pdf}
  }
\else                                            %PDFLaTeX
  \RequirePackage[pdftex,bookmarks=true]{hyperref}
  \hypersetup{ %
    pdfauthor   = {\@author},
    pdftitle    = {\@title},
    pdfcreator  = {LaTeX with hyperref package},
    pdfproducer = {dvips + ps2pdf}
  }
\pdfadjustspacing=1
\fi

% Set up counters for problems and subsections

\newcounter{ProblemNum}
\newcounter{SubProblemNum}[ProblemNum]

\renewcommand{\theProblemNum}{\arabic{ProblemNum}}
\renewcommand{\theSubProblemNum}{\alph{SubProblemNum}}

\newcommand*{\anyproblem}[1]{\newpage\subsection*{#1}}
\newcommand*{\problem}[1]{\stepcounter{ProblemNum} %
   \anyproblem{Problem \theProblemNum. \; #1}}
\newcommand*{\soln}[1]{\subsubsection*{#1}}
\newcommand*{\solution}{\soln{Solution}}
\renewcommand*{\part}{\stepcounter{SubProblemNum} %
  \soln{Part (\theSubProblemNum)}}

\renewcommand{\theenumi}{(\alph{enumi})}
\renewcommand{\labelenumi}{\theenumi}
\renewcommand{\theenumii}{\roman{enumii}}
\begin{document}

\problem{Decoding}

Consider the following set of codewords:

(A,B,C,D,E,F,G,H) = (01, 11, 001, 0000, 0001, 1001, 1010, 1011).

\begin{enumerate}
    \item Is this an instantaneous (prefix) code?
    \item Verify that it satisfies the Kraft inequality.
    \item Construct a string which has no meaning under this system Transmitting
        information across a channel.
\end{enumerate}

\solution

\part 

Yes. Function {\tt solvea} figures out if any code is a prefix of any other
code. The answer is that there is no code which prefix of any other code. Some
relevant snippet from the code is following.

\begin{verbatim}
*Main> alphabets 
"ABCDEFGH"
*Main> allcodes 
["01","11","001","0000","0001","1001","1010","1011"]
*Main> map isPrefixCode alphabets 
[False,False,False,False,False,False,False,False]
\end{verbatim}

\part

Function {\tt kraft } verifies that it is true. Following is the execution.

\begin{verbatim}
*Main> kraft_sum 
0.9375
*Main> kraft
True
*Main> 
\end{verbatim}

\part

Any string which function {\tt decoder} can't decode would have no meaning. Here
are some examples:

\begin{verbatim}
*Main> decode "101000101" 
"GCA"
*Main> decode "101000101111" 
"GCAB*** Exception: No valid alphabet for "1"
*Main> decode "1010001011110" 
"GCAB*** Exception: No valid alphabet for "10"
*Main> 
\end{verbatim}

\problem{} Consider the discrete memoryless channel Y = X + Z (mod 11), where Z
is uniformly distributed on the alphabet {1, 2, 3}, and $X \in \{0, 1, \ldots
10\}$ . Assume that Z is independent of X.

\begin{enumerate}
    \item Find the capacity.
    \item What is the maximizing $p^\star(x)$?
\end{enumerate}

\solution

\part During coding, any $x \in X$ goes to $x+1, x+2, x+3$ (mod 11) with
probability 1/3. This channel is very similar to noisy typewriter channel. Two
consider two possibilities, we can send either \{0, 3, 6, 9\} with a chance that
$9 \rightarrow 1$ and $0 \rightarrow 1$. Or we can drop 0 or 9 from the set. In
the former, we can send one of 4 symbols per transmission with a small
probability of error (i.e. $9 \rightarrow 1$ and $0 \rightarrow 1$); in later we
can send one of the 3 symbols per transmission without any error but we will
never send some symbols. None of these situations are desirable.

\footnote{Solve if one partition the scheme in a way that there is no
    collision}.

\problem{}

The Z channel has binary input and output alphabets and transition
probabilities $p(y|x)$ given by the following matrix: Q =

\begin{tabular}{c c c}
 & 1 &  0 \\
 &1/2 & 1/2 \\
\end{tabular}

$x, y \in \{0, 1\}$.

[Why is it called the Z-channel?]
Find the capacity of the Z channel and the maximizing input probability distribution.

\problem{}

[Optional] For the Z channel of the previous problem, assume that we choose a
($2^{nR}$) code at random, where each codeword is a sequence of fair coin
tosses. This will not achieve capacity. Find the maximum rate R such that the
probability of error $P_e^n$ , averaged over randomly generated codes, tends to
zero as the block length n tends to infinity.

\end{document}
