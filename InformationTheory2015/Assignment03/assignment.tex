\documentclass{article}
\usepackage[left=35mm,margin=20mm]{geometry}
\usepackage{ifthen,empheq}
\date{\today}
\usepackage{fancyhdr}
\usepackage[]{dot2texi}
\setlength{\headheight}{15pt}

\lhead{Assignment 1}
\chead{Dilawar Singh}
\rhead{\today}

\lfoot{}\cfoot{\thepage}\rfoot{}
\pagestyle{fancy}

\ifx\pdfoutput\undefined                         %LaTeX
  \RequirePackage[ps2pdf,bookmarks=true]{hyperref}
  \hypersetup{ %
    pdfauthor   = {\@author},
    pdftitle    = {\@title},
    pdfcreator  = {LaTeX with hyperref package},
    pdfproducer = {dvips + ps2pdf}
  }
\else                                            %PDFLaTeX
  \RequirePackage[pdftex,bookmarks=true]{hyperref}
  \hypersetup{ %
    pdfauthor   = {\@author},
    pdftitle    = {\@title},
    pdfcreator  = {LaTeX with hyperref package},
    pdfproducer = {dvips + ps2pdf}
  }
\pdfadjustspacing=1
\fi

% Set up counters for problems and subsections

\newcounter{ProblemNum}
\newcounter{SubProblemNum}[ProblemNum]

\renewcommand{\theProblemNum}{\arabic{ProblemNum}}
\renewcommand{\theSubProblemNum}{\alph{SubProblemNum}}

\newcommand*{\anyproblem}[1]{\newpage\subsection*{#1}}
\newcommand*{\problem}[1]{\stepcounter{ProblemNum} %
   \anyproblem{Problem \theProblemNum. \; #1}}
\newcommand*{\soln}[1]{\subsubsection*{#1}}
\newcommand*{\solution}{\soln{Solution}}
\renewcommand*{\part}{\stepcounter{SubProblemNum} %
  \soln{Part (\theSubProblemNum)}}

\renewcommand{\theenumi}{(\alph{enumi})}
\renewcommand{\labelenumi}{\theenumi}
\renewcommand{\theenumii}{\roman{enumii}}
\begin{document}

\problem{Counting Typical Sequences}

Being too stringent with the definition of typical sequences.  Consider a DNA
sequence of length 8 generated iid from the distribution 
    
$$ \wp(A, T, G, C) = (\frac{1}{2}, \frac{1}{4}, \frac{1}{8}, \frac{1}{8})$$

\part  

What is the single most probable sequence? What is its probability of
occurrence?

\solution 
The most probable sequence is $AAAAAAAA$ with probability of occurrence
$\frac{1}{2^8}$.

\part 
How many stringently typical sequences are there (exact answer required)?  
\solution
A (stringent) typical sequence $T_n$ is defined as
$-\frac{1}{n} log_2 p(T_n) - H(X) = 0$ which implies that $p(T_n) = 2^{-nH(x)}$.
One can calculate that $H(X) = 1.75$ and therefore $p(T_n) = 2^{-14}$. Since the
probability distribution has elements which are negative power of 2, the total
typical sequences are number of solutions to the equation $2^a + 2^b + 2^{3(c+d)} =
2^{14}$ where $a,b,c,d$ are positive integers. We can put further constrains 
$0 \ge a \le 14, 0 \ge b \le 7, 0 \ge c \le 4, 0 \ge d \le 4$. The snippet in
appedix solves this problem. There are total 2716 such sequences.

\part
What is the total probability of getting some stringently typical sequence?
\solution 
The probability of getting stringently typical sequence is
$\frac{2716}{4^8} = 0.04144$.

\part
Redo the whole calculation if the length is 16. What is the total probability
of getting a stringently typical sequence? Are we converging to 1?  

\solution
For $n=16$, total number of sequences are 31841628. The probability $p(T_n) =
0.007413$. For $n=4$, $p(T_n) = 0.09375$. We seem to be converging to 0.

\problem{Entropy rate of a Markov process}

The Morse code uses dots, dashes, and two types of spaces: letter and word
spaces.  Spaces only occur as delimiters; that is, no space can follow another
space. What is the entropy rate achievable under these constraints?  Hint: take
a look at Shannon’s 1948 paper for a start, and Example 5.1.3 of C\&T, but use
different notation. Write it out as a Markov process whose states are [.], [-],
[L-space], [W-space]. Assume that all allowed outputs from any state are equally
likely.  

\solution

The transition graph is shown below.

\begin{figure}[ht!]
\begin{center}
    \includegraphics[width=0.4\textwidth]{state.eps}
\end{center}
\caption{}
\label{fig:}
\end{figure}

This amounts to the following distribution where index 0, 1, 2, 3 maps to DOT,
DASH, LSPACE, WSSPACE.

\begin{verbatim}
[[ 0.25  0.25  0.5   0.5 ]
 [ 0.25  0.25  0.5   0.5 ]
 [ 0.25  0.25  0.    0.  ]
 [ 0.25  0.25  0.    0.  ]]
\end{verbatim}

One can compute the steady state distribution which is equivalent of finding
eigen vector with eigen value = 1. This value is [1/3, 1/3, 1/6, 1/6] after
normalization. The steady state distribution is following.

\begin{verbatim}
[[ 0.33333333  0.33333333  0.33333333  0.33333333]
 [ 0.33333333  0.33333333  0.33333333  0.33333333]
 [ 0.16666667  0.16666667  0.16666667  0.16666667]
 [ 0.16666667  0.16666667  0.16666667  0.16666667]]
\end{verbatim}

Given than any of these states are equally likely to begin with. We have 

\begin{verbatim}
Entropy rate: [[ 1.91829583]]
\end{verbatim}

See the python script at 

\problem{[Optional][C\&T 3.11]}

Read the C\&T section 4.3, Entropy Rate of a Random Walk.  What is the entropy
rate of rooks, kings, and bishops on a 3x3 chessboard?  Remember there are 2
types of bishops.

\end{document}


