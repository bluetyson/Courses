\documentclass[a4paper,10pt]{article}
\usepackage[margin=20mm, right=30mm]{geometry}
\usepackage{fancyhdr}
\usepackage{pgf,tikz}
\usepackage{amsmath}
\usepackage{url}
\usepackage[]{appendix}
\usepackage{amssymb}
\usepackage[]{algorithm2e}
\usetikzlibrary{shapes,backgrounds,decorations,decorations.pathmorphing}
\usepackage{listings}
% Title Page
\title{Assignment 1}
\author{Dilawar Singh}
\begin{document}
\maketitle


\section{Coding}

Consider a horse race where the probabilities of winning are given by negative
powers of 2 ($\frac{1}{2}, \frac{1}{32}$, etc.). Suppose the lowest possible
probability for any horse is $2^{-m}$ , and that there are $n_i$ horses in each
probability class $2^{-i}$.

\begin{enumerate}
    \item What equation must the $n_i$ satisfy?  
    \item Give 3 possible sets of $\{n_i\}$ for an 8 horse race (other than the
        uniform case).
    \item Using the entropy concept, calculate the expected codeword length for
        each set.
    \item For each set, present an efficient binary coding scheme that achieves
        optimality.
\end{enumerate}


\subsection{Part 1}
\label{sec:parta}

Since lowest probability of any horse winning is $\frac{1}{2^m} \implies i \leq
m$. Moreover, the probability of any horse winning a race must be 1 i.e. $
\sum_{i=1}^{i \leq m} \frac{1}{2^i} n_i = 1 $

\subsection{Part 2, 3}

Using the conditions described in section \ref{sec:parta}, we enumerated all
possible partitions of 8 horses for any given probabilities of winning
using listing \ref{listing:a}.


\begin{table}[h]
    \centering
\begin{tabular}{l l l}
    \hline
    probabilities ($p_1,p_2,p_3$)& partitions ($n_1,n_2,n_3$) & entropy \\
    \hline 
    (05,0.125,6.25e-2) & (1.0,1.0,6.0) & 2.375 \\
    (0.5,0.125,3.125e-2) & (1.0,3.0,4.0) & 2.25 \\
    (0.25,0.125,6.25e-2) & (1.0,5.0,2.0),(2.0,2.0,4.0) & 2.875,2.75 \\
    (0.25,0.125,3.125e-2) & (3.0,1.0,4.0) & 2.5 \\
    (0.25,6.25e-2,3.125e-2) & (3.0,3.0,2.0) & 2.5625 \\
    \hline
\end{tabular}
\caption{For a given probability of winning of each category $p_i$, right column
    shows all possible partition of 8 horses into these 3 categories e.g. first
    row says if the probability of winning of 3 categories are 0.5, 0.125, and
0.0625 then one can have 1, 1, 6 horses in these categories respectively.}
\end{table}

\subsection{Part 4}

To build an optimal code, we assume that optimality means least amount of
transfer of symbols over channel. If each horse has same probability of winning
then optimal coding scheme would be ${0, 1, 00, 01, 10, 11, 000, 001}$, with
average code length of $\frac{1+1+2+2+2+2+3+3}{8} = 2$.

We present the following algorithm to design a coding scheme and present a proof
that this algorithm generates an optimal coding scheme.

%% Here is algorithm
\begin{algorithm}[H]
    \KwData{$P = \{p(i): i \in H\}$ Probability of horse $i$ winning.}
    \KwResult{$H$, Code for each horse}
    $H \leftarrow sort(H)$ and $Q \leftarrow \{0,1,00,01,10,11,000,001,\ldots
    \}$\;

    \While{H is not empty}{
        horse $\leftarrow$ pop(H) \;
        $c \leftarrow pop(Q)$ \;
        assign c to horse.
    }
\end{algorithm}

Now we get the following coding scheme:

\begin{tabular}{c c c c c}
    \hline
    Probabilities & Possible partition & Entropy & Possible Coding & Avg
    code-len \\
    \hline
    1/2,1/8,1/16 & \{0\},\{1\},\{2,3,4,5,6,7\} & 2.375 &
    \{0\},\{1\},\{00,01,10,11,000,001\} & 1 \\
    1/2,1/8,1/32 & \{1\}, \{8,2,3\}, \{0,4,5,6,7\} & 2.25 &
    \{0\},\{1,00,01\},\{10,11,000,001\} & 1 \\
    1/4,1/8,1/32 & \{2,3,4\},\{0\},\{1,5,6,7\} & 2.5 & 
    \{0,1,00\},\{11\},\{01,10,000,111 \} & 1\\
    \hline
\end{tabular}


\begin{appendices}
    \section{Haskell listing}
    \label{listing:a}

    \lstinputlisting[language=Haskell
    ,basicstyle=\scriptsize\ttfamily]{./assign1.hs}

\end{appendices}

\end{document}          
