\documentclass[]{article}
\usepackage[T1]{fontenc}
\usepackage{lmodern}
%\usepackage{circuitikz}
\makeatletter
\@ifpackageloaded{tex4ht}{
    \def\pgfsysdriver{pgfsys-tex4ht.def}
}
\makeatother
\usepackage{tikz}
\usepackage{siunitx}
\usepackage{python}
\usepackage{pgfplots}
\usepackage{amssymb,amsmath}
\usepackage{ifxetex,ifluatex}
\usepackage{amsthm}
\usepackage{listings}
\setlength{\parskip}{3mm}
\newtheorem{axiom}{Axiom}
\newtheorem{definition}{Definition}
\newtheorem{comment}{Comment}
\newtheorem{example}{Example}
\newtheorem{lemma}{Lemma}
\newtheorem{property}{Property}
\newtheorem{problem}{Problem}
\newtheorem{remark}{Remark}
\newtheorem{theorem}{Theorem}
\newtheorem{script}{Script}

\usepackage{fixltx2e} % provides \textsubscript
% use upquote if available, for straight quotes in verbatim environments
\IfFileExists{upquote.sty}{\usepackage{upquote}}{}
\ifnum 0\ifxetex 1\fi\ifluatex 1\fi=0 % if pdftex
  \usepackage[utf8]{inputenc}
\else % if luatex or xelatex
  \ifxetex
    \usepackage{mathspec}
    \usepackage{xltxtra,xunicode}
  \else
    \usepackage{fontspec}
  \fi
  \defaultfontfeatures{Mapping=tex-text,Scale=MatchLowercase}
  \newcommand{\euro}{€}
\fi
% use microtype if available
\IfFileExists{microtype.sty}{\usepackage{microtype}}{}
\ifxetex
  \usepackage[setpagesize=false, % page size defined by xetex
              unicode=false, % unicode breaks when used with xetex
              xetex]{hyperref}
\else
  \usepackage[unicode=true]{hyperref}
\fi
\hypersetup{breaklinks=true,
            bookmarks=true,
            pdfauthor={Dilawar Singh},
            pdftitle={Problem 1},
            colorlinks=true,
            citecolor=blue,
            urlcolor=blue,
            linkcolor=magenta,
            pdfborder={0 0 0}}
\urlstyle{same}  % don't use monospace font for urls
\setlength{\parindent}{0pt}
\setlength{\parskip}{6pt plus 2pt minus 1pt}
\setlength{\emergencystretch}{3em}  % prevent overfull lines
\setcounter{secnumdepth}{0}

\title{Problem 1}
\author{Dilawar Singh}
\date{A pandoc markdown version of this document is available
\href{http://github.com/dilawar/courses/raw/master/NeuroCourse/Assignments/Second/Problem01/L2_Q1_DilawarSingh.pandoc}{here}}

\begin{document}
\maketitle

\begin{problem}
 The axon from your spinal cord to your toe is about 1 metre long. How fat
 would it have to be to conduct stimuli, assuming that the potential at the
 stimulus is 0 mV and the potential at the terminus is -38 mV. Cell properties
 are: Resting potential = -60 mV, $R_A$ = 1 ohm metre, and $R_M$ = 1 ohm $m^2$

 \footnote{Circuit drawn using pgf \pgfversion}

\end{problem}

Since we are dealing with the steady-state value at output, we need not
consider the transient elements. The equivalent circuit of cell will be
like a potential-divider. To make our life simpler, we raise the
potential of every node by 60 mV. os

\begin{figure}[h!]
\usetikzlibrary{circuits.ee.IEC}
\usetikzlibrary{shapes}
\centering
\begin{tikzpicture}[circuit ee IEC
    , small circuit symbols
    , set resistor graphic=var resistor IEC graphic
    ]
\node [cylinder
    , cylinder uses custom fill
    , minimum width=1.5cm, minimum height=12cm
    , cylinder end fill=red!50, cylinder body fill=red!25] (c)  
    at (6,0) {};

    \node (a) at (c.west) [left] {A};
    \node (b) at (c.east) [right] {B};
    \draw[<->] (c.after bottom) -- (c.north) node[midway, above] {x};
    \draw[<->] (c.north) -- +(1.5,0) node[midway, above] {dx};

%% Draw axial resitors 
\foreach \x in {0,1.5,...,10.5}
{
    \draw (\x,0) -- +(0.25,0) to[resistor={info={$dR$}}] (\x+1.25,0) -- +(0.25,0);
    \draw (\x+1.5,0) to[resistor={info={$r_m$}}] (\x+1.5,-1.5) to[ground] +(0,-0.25);
}
\end{tikzpicture}

\caption{\small Equivalent cable approximation of axon at steady state. To make
    our analysis simpler, we raise the potential of every node by 60 mV.
    Therefore, end A is at 60 mV and end B is at 22 mV. 
}
\end{figure}

\subsection{Circuit analysis of this network at steady
state}\label{circuit-analysis-of-this-network-at-steady-state}

Network at steady state i.e.~node potential are changing in time. We can
do nodal analysis now. Take any three consecutive nodes. For
convenience, we chose following locations: $x-dx$, $x$, and $x+dx$. We
get the following equations.

\begin{align}
\frac{V(x) - V(x-dx)}{dR} + \frac{V(x) - V(x+dx)}{dR} + \frac{V(x)}{r_m} &= 0
\end{align}

We also have following relations: $dR=\frac{R_A dx}{\pi r^2}$ where $r$
is the radius of axon, also, $r_m =\frac{R_M}{2\pi r dx}$. After putting
them all together, we get.

\begin{align}
\frac{V(x) - V(x-dx)}{\frac{R_A dx}{\pi r^2}} + \frac{V(x) - V(x+dx)}
    {\frac{R_A dx}{\pi r^2}} + \frac{V(x)}{\frac{R_M}{2\pi r dx}} &= 0  \\
\frac{\pi r^2}{R_A} \left( \frac{V(x) - V(x-dx)}{dx} +
    \frac{V(x) - V(x+dx)}{dx} + \frac{V(x)}{\frac{R_Mr}{R_A 2dx}} \right) &= 0 \\ 
\frac{1}{dx} \left( \frac{V(x) - V(x-dx)}{dx} + \frac{V(x) - V(x+dx)}{dx} \right)
    &= - \frac{V(x)}{\frac{R_Mr}{2R_A}} 
\end{align}

Some more rearranging gives us the following/

\begin{align}
\frac{1}{dx} \left( 
    \frac{V(x+dx) - V(x)}{dx} - \frac{V(x) - V(x-dx)}{dx} 
\right) &=  \frac{V(x)}{\frac{R_Mr}{2R_A}} 
\end{align}

One left hand side, we have the definition of second derivative. Now we
can claim that life is easy.

\begin{eqnarray}
    \label{eqn}
    \frac{d^2V(x)}{dx^2} &= \frac{V(x)}{\frac{R_Mr}{2R_A}}
\end{eqnarray}

Let's put the values.

\begin{eqnarray}
    \frac{d^2V(x)}{dx^2} &= \frac{2}{r}  V(x)
\end{eqnarray}

\subsection{Solution}\label{solution}

The above equation has a solution $V(x) = c \exp \sqrt{\frac{2}{r}}x$
Now we solve it for given boundry conditions: $V(0) = 60$ and
$V(1) = 22$. This gives us the following solution:
$\frac{2}{r}=  [\ln \frac{22}{60}]^2 \implies r = 1.9868 m$. Such a
thick axon!

\end{document}
