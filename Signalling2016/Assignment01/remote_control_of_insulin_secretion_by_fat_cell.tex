\documentclass[]{article}
\usepackage{mathtools,amsmath,amssymb,amsthm,mathrsfs}
\usepackage[utf8]{inputenc}
\usepackage{lmodern}
\usepackage{tikz}
\usetikzlibrary{graphs}
\usetikzlibrary{calc}
\usetikzlibrary{graphdrawing}
\usetikzlibrary{quotes}
\usegdlibrary{layered}
\usetikzlibrary{arrows}
\makeatletter
\title{Assignment 1}
\author{Dilawar Singh}
\date{\today}

\begin{document}
\maketitle

Molecules are shown in black, all reactions/processes are in blue. A label "x/y"
mean that molecule "x" is in region/place/cell "y".

\section{The secretion of brain dilps is controlled by diet/amino acids}
\label{the-secretion-of-brain-dilps-is-controlled-by-diet}

\begin{figure}[ht!]
    \centering
\begin{tikzpicture}[scale=1
    , every node/.style={}
    ]

    \graph[ layered layout,  nodes = {  } 
        ]
    {
        "Insulin Producing Cells" -> secretion[blue] -> "Dilps/IPC"
        -> circulation[blue] -> "Dlips/hemolymph";

        feeding[blue] -> nutrients;
        %starvation[blue] ->[-*] nutrients;
        nutrients ->[-*] circulation;

    };
\end{tikzpicture}    
\caption{
    Dilps accumulates in IPCs when larva is starved which is reverted
        by refeeding (too fast (15 minutes) to be controlled by increasing gene
        expression). Starvation of amino acids is sufficient to have this response.
    } 
\end{figure}

\footnote{Does lack of "feeding" increases the "secretion" with or without
    affecting the rate of circulation?}

\section{The brain IPCs couple neurosecretion with nutritional inputs}


\section{Notes}\label{notes}

\begin{itemize}
\item
  The Drosophila \emph{Insulin/Insuling like growth factor signalling}
  (strangely abbrebriated IIS) is similar to its human counterpart.
\item
\end{itemize}

\end{document}
