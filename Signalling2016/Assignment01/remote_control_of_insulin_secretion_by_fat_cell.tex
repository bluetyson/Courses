\documentclass[]{article}
\usepackage{mathtools,amsmath,amssymb,amsthm,mathrsfs}
\usepackage[utf8]{inputenc}
\usepackage{lmodern}
\usepackage{tikz}
\usetikzlibrary{graphs}
\usetikzlibrary{calc}
\usetikzlibrary{graphdrawing}
\usetikzlibrary{quotes}
\usegdlibrary{layered}
\usetikzlibrary{arrows}
\makeatletter
\title{Assignment 1}
\author{Dilawar Singh}
\date{\today}

\begin{document}
\maketitle

Molecules are shown in black, all reactions/processes are in blue. A label "x/y"
mean that molecule "x" is in region/place/cell "y".

\section{The secretion of brain dilps is controlled by diet/amino acids}
\label{the-secretion-of-brain-dilps-is-controlled-by-diet}

\begin{figure}[ht!]
    \centering
\begin{tikzpicture}[scale=1
    , every node/.style={}
    ]

    \graph[ layered layout,  nodes = {  } 
        , edges = { inner sep = 1pt, thick }
        ]
    {
        "Insulin Producing Cells" -> secretion[blue] -> "Dilps/IPC"
        -> circulation[blue] -> "Dilps/hemolymph";

        feeding[blue] -> nutrients;
        %starvation[blue] ->[-*] nutrients;
        nutrients ->[-|] circulation;

    };
\end{tikzpicture}    
\caption{
    Dilps accumulates in IPCs when larva is starved which is reverted
        by refeeding (too fast (15 minutes) to be controlled by increasing gene
        expression). Starvation of amino acids is sufficient to have this response.
    } 
\end{figure}

\footnote{Does lack of "feeding" increases the "secretion" with or without
    affecting the rate of circulation?}

\section{The brain IPCs couple neurosecretion with nutritional inputs}

\begin{figure}[h]
    \centering
    \begin{tikzpicture}[scale=1 , every node/.style={} ]
        \graph [ layered layout, nodes = { }, edge = { thick } ]
        {
            "release from IPC"[blue]; 
            secretion[blue];

            nutrients -> "???"[blue] -> secretion;
            
            "hyperpolarization/Kir2.1" [blue] ->[-|] "release from IPC";

            secretion [ blue] -> "Dilps/IPC" -> "release from IPC" -> "Dilps/hemolymph"
                -> growth [blue];

            TOR ->[-*] "release from IPC";

            %% {[same layer] TOR, "release from IPC" };
        };

    \end{tikzpicture}    
    \caption{
        Kir2.1 is a potassium channel. Overexpression of Dilps in IPC have very
        little incrementally effect on groth ( $\implies$ Dilps level in IPC
            did not change much ). Membrane potential is important for release
        of Dilps from cell body.  
    } 

    \end{figure}

\section{The fat body remotely controls insulin release from brain IPCs}


\footnote{ Don't know if ??? downregulates TOR, IIS and growth in parallel or
    some other topology exists e.g. ???? downregulates TOR which in turn
    downregulates IIS etc. }

\begin{figure}[h]
\begin{center}

    \begin{tikzpicture}[scale=1
        , every node/.style={}
        ]
        
        \graph[ layered layout, nodes = {}, edges = { thick } ]
        {

            "slif expr knockdown"[blue] ->[-|] "amino acids" 
            -> "????"[blue] ->[-|] { IIS, growth }; 

            "amino acids" -> TOR ->[-*] "release from IPC"[blue];
        };

    \end{tikzpicture}    
\end{center}
\caption{}
\label{fig:}
\end{figure}

\section{Direct humoral link between the fat body and the brain}

\begin{figure}[h!]
    \centering
    \begin{tikzpicture}[scale=1
        , every node/.style={}
        ]

        \graph [ nodes = {}
            , edge = { thick }
            , layered layout 
            ]
        {
            release[blue];
            "Dilps/IPC" -> release -> "Dilps/hemolymph";

            % Adding fat bodies from starved animals did not change the
            % concentration of Dilps in brain.
            "starved fat-bodies" --[thin,dotted] release;

            % However adding fat bodies from well fed animal reduce the
            % concentration of Dilps in brain;
            "fed fat-bodies" ->[-*] release;

            "KCl/depolarization of IPC cell" [blue] ->[-*] release;

            %{ [same layer] "starved fat-bodies", "fed fat-bodies" };
        }
    \end{tikzpicture}    
\label{fig:}
\end{figure}

\section{Summary}

\begin{figure}[h!]
    \centering
    \begin{tikzpicture}[scale=1 , every node/.style={} ]
        \graph [ edges = { thick }, nodes = {}, layered layout ]
        {
            "Insulin Producing Cells" -> secretion[blue] -> "Dilps/IPC"
            -> release[blue] -> "Dilps/hemolymph";

            feeding[blue] -> nutrients;
            %starvation[blue] ->[-*] nutrients;
            nutrients -> "fat body" -> "FactorX/hemolymph" -> "???"[blue] -> release;

            "FactorX/hemolymph" ->[dotted,thin, "??"] "hyperpolarization";

            secretion[blue];


            "hyperpolarization" [blue] ->[-|] "release";

            secretion [ blue] -> "Dilps/IPC" -> "release" -> "Dilps/hemolymph"
            -> growth [blue];

            TOR ->[-*] "release";
        }
    \end{tikzpicture}    
\end{figure}


%% NOTES
\section{Notes}\label{notes}
\begin{itemize}
\item
  The Drosophila \emph{Insulin/Insuling like growth factor signalling}
  (strangely abbrebriated IIS) is similar to its human counterpart.
\item Suggest that "FactorX" is a secretion factor. 
\end{itemize}

\end{document}
