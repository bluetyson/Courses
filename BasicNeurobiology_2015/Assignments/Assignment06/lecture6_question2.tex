\documentclass[]{article}
\usepackage{amsmath,amssymb,amsthm}
\usepackage[utf8]{inputenc}
\usepackage{lmodern}
%\usepackage{circuitikz}
\makeatletter
\@ifpackageloaded{tex4ht}{
    \def\pgfsysdriver{pgfsys-tex4ht.def}
}
\makeatother
\usepackage{pgfplots}
\usepackage{pgfplotstable}
\usepackage{pgf,tikz}
\usetikzlibrary{shapes,backgrounds,positioning,matrix,decorations}

\usepackage{siunitx}
\usepackage{python}
\usepackage{ifxetex,ifluatex}
\usepackage{listings}
% \usepackage[xindy,acronym,nomain,toc]{glossaries}
% \makeglossaries
%\usepackage[xindy]{imakeidx}
%\makeindex
\setlength{\parskip}{3mm}
\newtheorem{axiom}{Axiom}
\newtheorem{definition}{Definition}
\newtheorem{comment}{Comment}
\newtheorem{example}{Example}
\newtheorem{lemma}{Lemma}
\newtheorem{property}{Property}
\newtheorem{problem}{Problem}
\newtheorem{remark}{Remark}
\newtheorem{theorem}{Theorem}
\newtheorem{script}{Script}

\usepackage{fixltx2e} % provides \textsubscript
% use upquote if available, for straight quotes in verbatim environments
\IfFileExists{upquote.sty}{\usepackage{upquote}}{}
\ifnum 0\ifxetex 1\fi\ifluatex 1\fi=0 % if pdftex
  \usepackage[utf8]{inputenc}
\else % if luatex or xelatex
  \ifxetex
    \usepackage{mathspec}
    \usepackage{xltxtra,xunicode}
  \else
    \usepackage{fontspec}
  \fi
  \defaultfontfeatures{Mapping=tex-text,Scale=MatchLowercase}
  \newcommand{\euro}{€}
\fi
% use microtype if available
\IfFileExists{microtype.sty}{\usepackage{microtype}}{}
\ifxetex
  \usepackage[setpagesize=false, % page size defined by xetex
              unicode=false, % unicode breaks when used with xetex
              xetex]{hyperref}
\else
  \usepackage[unicode=true]{hyperref}
\fi
\hypersetup{breaklinks=true,
            bookmarks=true,
            pdfauthor={Dilawar Singh},
            pdftitle={Lecture 6, Problem 2},
            colorlinks=true,
            citecolor=blue,
            urlcolor=blue,
            linkcolor=magenta,
            pdfborder={0 0 0}}
\urlstyle{same}  % don't use monospace font for urls
\setlength{\parindent}{0pt}
\setlength{\parskip}{6pt plus 2pt minus 1pt}
\setlength{\emergencystretch}{3em}  % prevent overfull lines
\setcounter{secnumdepth}{5}

\title{Lecture 6, Problem 2}
\author{Dilawar Singh}
\date{\today}
\usepackage[margin=15mm]{geometry}

\begin{document}
\maketitle

\begin{quote}
I want to encode colour using a distributed code where I only have two
sensors, red and blue. How will the system respond if I give it just a
single wavelength in the green? Can you give this system a combination
of other wavelengths that it will confuse with green? How does the
provision of 3 sensors avoid this?
\end{quote}

With 2 sensors X and Y, I can encode 4 values: 00, 01, 10 and 11 where
00 means X off and Y off, 01 means X off and Y on and so on. Green light
will not cause any of the sensor to be on, producing 00 (which is
essentially detection of no color). If I turn both switches (combination
of not Blue and not Red), again it will emit 00 and confuse it with
green. Essentially any combination with no Red and no Blue in it will
produce 00 (which we mapped to green).

Having three sensors allows one to detect the presence or absence of
green as well. Total 8 values can be encoded now where 000 means no Red,
Blue or Green present, and 111 means and all three colors present.

\end{document}
