\documentclass[]{article}
\usepackage{amsmath,amssymb,amsthm}
\usepackage[utf8]{inputenc}
\usepackage{lmodern}
%\usepackage{circuitikz}
\makeatletter
\@ifpackageloaded{tex4ht}{
    \def\pgfsysdriver{pgfsys-tex4ht.def}
}
\makeatother
\usepackage{pgfplots}
\usepackage{pgfplotstable}
\usepackage{pgf,tikz}
\usetikzlibrary{shapes,backgrounds,positioning,matrix,decorations}

\usepackage{siunitx}
\usepackage{python}
\usepackage{ifxetex,ifluatex}
\usepackage{listings}
% \usepackage[xindy,acronym,nomain,toc]{glossaries}
% \makeglossaries
%\usepackage[xindy]{imakeidx}
%\makeindex
\setlength{\parskip}{3mm}
\newtheorem{axiom}{Axiom}
\newtheorem{definition}{Definition}
\newtheorem{comment}{Comment}
\newtheorem{example}{Example}
\newtheorem{lemma}{Lemma}
\newtheorem{property}{Property}
\newtheorem{problem}{Problem}
\newtheorem{remark}{Remark}
\newtheorem{theorem}{Theorem}
\newtheorem{script}{Script}

\usepackage{fixltx2e} % provides \textsubscript
% use upquote if available, for straight quotes in verbatim environments
\IfFileExists{upquote.sty}{\usepackage{upquote}}{}
\ifnum 0\ifxetex 1\fi\ifluatex 1\fi=0 % if pdftex
  \usepackage[utf8]{inputenc}
\else % if luatex or xelatex
  \ifxetex
    \usepackage{mathspec}
    \usepackage{xltxtra,xunicode}
  \else
    \usepackage{fontspec}
  \fi
  \defaultfontfeatures{Mapping=tex-text,Scale=MatchLowercase}
  \newcommand{\euro}{€}
\fi
% use microtype if available
\IfFileExists{microtype.sty}{\usepackage{microtype}}{}
\ifxetex
  \usepackage[setpagesize=false, % page size defined by xetex
              unicode=false, % unicode breaks when used with xetex
              xetex]{hyperref}
\else
  \usepackage[unicode=true]{hyperref}
\fi
\hypersetup{breaklinks=true,
            bookmarks=true,
            pdfauthor={Dilawar Singh},
            pdftitle={Lecture 5, Question 1},
            colorlinks=true,
            citecolor=blue,
            urlcolor=blue,
            linkcolor=magenta,
            pdfborder={0 0 0}}
\urlstyle{same}  % don't use monospace font for urls
\setlength{\parindent}{0pt}
\setlength{\parskip}{6pt plus 2pt minus 1pt}
\setlength{\emergencystretch}{3em}  % prevent overfull lines
\setcounter{secnumdepth}{5}

\title{Lecture 5, Question 1}
\author{Dilawar Singh}
\date{\today}
\usepackage{circuitikz}

\begin{document}
\maketitle

\begin{quote}
I change the extracellular solution on a squid axon so it has 50 mM K+
ions.

\begin{enumerate}
\def\labelenumi{\alph{enumi})}
\item
  What is the new reversal potential of the K channel?
\item
  Assume that the previous resting potential of -65 mV was due to a leak
  at zero mV through Rm, in parallel with the previous reversal
  potential of the K channel of about -75 mV. What is the ratio of Rm to
  the resting conductance of the K channel?\\
\item
  What is the new resting potential (Vm) of the cell, assuming that the
  conductance of the K channel and Rm are still in the same ratio?
\end{enumerate}
\end{quote}

\textbf{Solution a}

Nernst equation at 25 deg C.

\[ E = \frac{59.3}{z} \log{\frac{[out]}{[in]}} mV \]

Which gives us:

\[ 59.3 \times \log{\frac{50}{400}} = -54.54 mV \]

\textbf{Solution b}

Effective circuit diagram in this case looks like the following. There
is loop current $i$ which causes voltage drop in $R_K$ to bring the
membrane potential down to $-65 mV$. We write the node-equations.

\begin{circuitikz}
\centering
\draw (0,0) to [R=$R_m$] (0, -3);
\draw (0,-3) to[short,*-*] (3, -3) to [battery1=$E_K$] (3, -2) to[R=$R_K$] (3, 0)
to[short,*-*] (0,0);
\node[scale=3] (a) at (1.5, -1.5) {$\circlearrowleft$} node[] at (1.5,-1.5) {$i$};
    
\end{circuitikz}

\[ (-65 + 75) g_K = - i \] \[ -65 g_m =  i \]

Therefore,
$10 g_K = 65 g_m \implies \frac{g_K}{g_m} = \frac{R_m}{R_K} = 6.5$.

\textbf{Solution c}

Putting new value of $E_K$ in our equation, we get

\[ \frac{V_m + 54.5}{R_K} + \frac{V_m}{R_m} = 0\]
\[ V_m = -54.54 \frac{R_m}{R_m + R_K} = -47.27 mV \]

\end{document}
