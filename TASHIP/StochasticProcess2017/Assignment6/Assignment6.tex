%%=====================================================================================
%%
%%       Filename:  Assignment6.tex
%%
%%    Description:  Solution to assignment 6.
%%
%%        Version:  1.0
%%        Created:  03/28/2017
%%       Revision:  none
%%
%%         Author:  Dilawar Singh (), dilawars@ncbs.res.in
%%   Organization:  NCBS Bangalore
%%      Copyright:  Copyright (c) 2017, Dilawar Singh
%%
%%          Notes:  
%%                
%%=====================================================================================

\documentclass[a4paper,10pt]{article}
\usepackage[margin=20mm]{geometry}
\usepackage{pgf,tikz}
\usepackage{amsmath}
\usepackage{amssymb}
\usepackage{verbatim}
\usepackage{hyperref}
\usepackage{pgfplots}
\usetikzlibrary{shapes,backgrounds,decorations,decorations.pathmorphing}

% externalize tikz graphs
\usetikzlibrary{external}
\tikzexternalize
    
% Title Page
\title{Solution : Homework 6} 
\author{Dilawar Singh}
\date{\today}
\begin{document}
\maketitle

The simulation is done in Haskell;  Since Haskell is not a very mainstream
language, code is in appendix for your reference only. It is available on
demand. It can also be downloaded 
\href{https://raw.githubusercontent.com/dilawar/Courses/master/TASHIP/StochasticProcess2017/Assignment6/solution.hs}{from
github}.

\section{Solution}

For each case we plot 5 sample trajectories.

\foreach \vel in {0,10}
{
    \foreach \dt in {0.1, 0.01, 0.001}
    {
        \pgfmathsetmacro\everyN{round(1.0/\dt)}
        \begin{tikzpicture}[scale=1]
            \begin{axis}[
                    xlabel=Time (au) ,ylabel= Position (au)
                    , grid style={draw=gray!20}
                    , grid = both
                    , title =  $v_0$ \vel\; and dt \dt
                    , width = 3cm, scale only axis
                    , no marks
                    ]
                \foreach \i in {1,2,3,4,5}
                {
                    \addplot  gnuplot [ raw gnuplot ] {
                        set datafile separator ",";
                        plot "./_data/\vel _\dt _\i .txt" using 1:3 every \everyN with lines;
                    };
                }
            \end{axis}
        \end{tikzpicture}
    }
    \par
}

\subsection*{Part B}

Next, we plot the histogram if velocity. Is it possible to tell if all of them
are same? By the look from it, they look the same. Usually one do statistical
test (e.g. KS test) to verify.


\par \noindent

\begin{figure}[ht!]
\begin{center}
\foreach \vel in {0,10}
{
    \foreach \dt in {0.1, 0.01, 0.001}
    {
        \pgfmathsetmacro\everyN{round(1.0/\dt)}
        \edef\datafile{\vel _\dt _lastVel.txt}
        \begin{tikzpicture}[scale=1]
            \begin{axis}[ title =  $v_0$ \vel\; and dt \dt
                    , width = 4cm, scale only axis 
                    %, xmin = -4, xmax = 4
                ]
                \addplot[ybar, bar width=5,fill=red,draw=none] gnuplot [raw gnuplot ] {
                    set datafile separator ",";
                    bin(x,width)=width*floor(x/width);
                    plot "\datafile" using (bin($2,0.5)):(1.0) smooth freq with boxes;
                };

                \addplot[mark=x ] gnuplot [ raw gnuplot ] {
                    set datafile separator ",";
                    stat "\datafile" using 2 name "A"; 
                    plot "+" using ((A_sumsq / A_records) ** 0.5):(0) w p;
                };

            \end{axis}
        \end{tikzpicture} %
    }
    \par
}

\end{center}
\caption{
Histogram of velocity. The black cross is the root-mean-squared value
of velocity as 100 units of simulation. The histogram look similar for given
number of entries. 
}
\label{fig:}
\end{figure}

\subsection*{Bonus}

We use the same data to compute the diffusion coefficient using 
$(x_t - x_0)^2 = 2Dt$. After some time, the slope is getting less variable and
seems to be getting constant.

\begin{tikzpicture}[scale=1]
    \begin{axis}[ xlabel=Time, ylabel = 2Dt
        , grid style={draw=gray!20}, grid = both
    ]
    \addplot+ gnuplot [ raw gnuplot ] {
        set datafile separator ",";
        plot "./diff_results.csv" using 1:2 every 100 with lines
    };
    \end{axis}
\end{tikzpicture}

\newpage
\appendix
\section{Haskell based simulator}
\verbatiminput{./solution.hs}

\end{document}          
