%%=====================================================================================
%%
%%       Filename:  solution1b.tex
%%
%%    Description:  Solution to problem 1 c
%%
%%        Version:  1.0
%%        Created:  03/10/2017
%%       Revision:  none
%%
%%         Author:  Dilawar Singh (), dilawars@ncbs.res.in
%%   Organization:  NCBS Bangalore
%%      Copyright:  Copyright (c) 2017, Dilawar Singh
%%
%%          Notes:  
%%                
%%=====================================================================================

\documentclass[a4paper,10pt]{article}
\usepackage{pgf,tikz}
\usepackage{amsmath}
\usepackage{amssymb}
\usepackage{circuitikz}
\usepackage{pgfplots}
\usetikzlibrary{calc,shapes,arrows,positioning,arrows.meta}
\usetikzlibrary{shapes,backgrounds,decorations,decorations.pathmorphing}

% Title Page
\title{Solution to problem 1b}
\author{Dilawar Singh}
\date{\today}
\begin{document}
\maketitle

\paragraph{Problem statement}
The NCBS lightning conductors used to run horizontally along the tops of the
stone railings near the Simons Centre. There was one lightning  rod above the
Simons Centre, and another near the place where the greenhouse now stands, and
yet another further along. This was a good arrangement to ensure that wherever
you were, you would die if lightning hit any part of the building. Draw the
equivalent circuit of the lightning circuit, assuming that each rod has a
straight wire going to ground, and a horizontal wire of the same length and
diameter going to the nearest lightning rod. The middle rod will therefore be
connected to ground by one wire, and by horizontal wires to the other two rods.
Assume that the potential is 1 million volts at one rod, where the lightning has
just struck. What is the potential at the middle rod?

\paragraph{Solution} 
\label{par:Solution}
Below is the equivalent circuit. You can solve for the voltage at middle node
(Greenhouse).

\begin{figure}[h!]
\centering
\begin{circuitikz}[ ]
    \draw (0,0) to [R=$R_1$] (0,-2) node [ground] {};
    \draw (0,0) node[above,rotate=90,xshift=1cm] {Simon} to [R=$R_2$, *-*] (2,0);
    \draw (2,0) to [R=$R_1$] (2,-2) node [ground] {};
    \draw (2,0) node[above,rotate=90,xshift=1cm] {Greenhouse} to [R=$R_2$, *-*] (4,0);
    \draw (4,0) node[above,rotate=90,xshift=1cm] {} to [R=$R_1$] (4,-2) node [ground] {};
    \draw (5.5,1.5) node[ground,rotate=135] {} to[pvsource=$10^6$V] (4,0);
    \draw (1,-1) node[scale=2] {$\circlearrowright$};
    \draw (1,-1) node[] {$i_1$};

    \draw (3,-1) node[scale=2] {$\circlearrowright$};
    \draw (3,-1) node[] {$i_2$};

    \draw (5,-1) node[scale=2] {$\circlearrowright$};
    \draw (5,-1) node[] {$i_3$};
    
\end{circuitikz}

\caption{\small Lightening rods in NCBS}
\label{circuit:label}
\end{figure}

We can reduce the circuit to simplify the calculations or write down the
Kirchhoff loop current equations. Let's take the hard way out.

\begin{align}
    i_1 R_1 + i_2 R_2 + (i_1 - i_2) R_1 &= 0  \\
    (i_2 - i_1) R_1 + i_2 R_2 + (i_2 - i_3) R_1 &= 0  \\
    ( i_3 - i_2) R_1 &= 10^6
\end{align}

When you solve these equations, you get:

\begin{align}
    i_1 &= \frac{10^6 R_1}{R_1^2 + 3R_1R_2 + R_2^2} \\
    i_2 &= \frac{10^6 (2R_1 + R_2)}{R_1^2 + 3R_1R_2+R_2^2} 
\end{align}

We are looking for voltage at middle rod, which is $(i_2 - i_1)R_1$. It is 
$10^6 \frac{R_1^2 + R_1R_2}{R_1^2 + R_2^2 + 3R_1R_2}$. When $R_1 = R_2$, voltage is $\frac{2}{5}$
times of 1 million. Let's see how this V behaves as a function of
$\frac{R_2}{R_1}$.

\begin{tikzpicture}[scale=1]
    \begin{semilogxaxis}[
            xlabel=$\frac{R_2}{R_1}$,ylabel=Voltage at middle node
            , grid style={draw=gray!20}, grid = both, minor tick num = 4 
    ]
    \addplot [color=blue] gnuplot [ raw gnuplot ] {
            f(x)=10^6*(1+x)/(1+x^2+3*x);
            set logscale x;
            set xrange [1:1000];
            plot f(x);
    };
    \end{semilogxaxis}
\end{tikzpicture}

Even when value of $R_2$ is 1000 times $R_1$, the voltage at middle node can be
dangerously high (998.0 Volts) - enough to kill!



\end{document}          
