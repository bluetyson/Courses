%%=====================================================================================
%%
%%       Filename:  solution_l4_q456.tex
%%
%%    Description:  Solution to problems 4, 5 and 6 (lecture 4).
%%
%%        Version:  1.0
%%        Created:  03/10/2017
%%       Revision:  none
%%
%%         Author:  Dilawar Singh (), dilawars@ncbs.res.in
%%   Organization:  NCBS Bangalore
%%      Copyright:  Copyright (c) 2017, Dilawar Singh
%%
%%          Notes:  
%%                
%%=====================================================================================

\documentclass[a4paper,10pt]{article}
\usepackage{pgf,tikz}
\usepackage{pgfplots}
\usepackage{amsmath}
\usepackage{amssymb}
\usetikzlibrary{shapes,backgrounds,decorations,decorations.pathmorphing}
% Title Page
\title{Lecture 4, Question 4,5, and 6} 
\author{Dilawar Singh}
\date{\today}
\begin{document}
\maketitle

\paragraph{Problem 4 - 5 Points}
\label{par:Problem 4}
What would happen if you increased the frequency of square wave current input to 
an equivalent circuit of a neuron? Draw the waveform at 1, 10, 100 and 1000 Hz, 
explaining.

\paragraph{Solution} Let's start with our single compartment neuron (notice that
$R_a$ is not playing any role in this circuit).

\begin{figure}[h]
    \usetikzlibrary{shapes,shadows,circuits.ee.IEC}
    \centering
    \begin{tikzpicture}[circuit ee IEC
        ]
        % Connect R_a
        \draw (-3,0) to[resistor={info={$R_a$}}] ++(3,0)
        ;
        \draw (0,0) -- ++(0,-0.5) -- ++(-1,0)
        to[resistor={info={$R_m$}}] ++(0,-2) 
        to[voltage source={direction info={<-,volt=0.06}}] ++(0,-1.5)
        ;
        \draw (0,0) -- ++(0,-0.5) -- ++(1,0)
        to[capacitor={info={$C_m$}}] ++(0,-3.5) -- ++(-2,0)
        ;
        \draw (0,-4.0) -- ++(0,-0.5) to[ground] ++(0,-0.5)
        ;
        %% Connect a probe to inject current 
        \draw  (2,1) to[current source={direction info={->,ampere=10p}}] ++(-1,0)
            -- ++(-1,-1) node [above] {$b$}
        ;
        \draw (2,1) -- ++(1,0) to[ground] ++(0.5,0)
        ;

    \end{tikzpicture}
    \caption{Injecting current into cell. Axial resistance $R_a$ is useless now.
    We measure voltage at node $b$ when current is being injected into cell.}
    \label{fig:sphere}
\end{figure}

For some typical values of $R_m$, $C_m$, I solve this circuit and plot the
waveforms for given frequencies. When I plot them together, I see the following:

\def\height{6cm}
\begin{tikzpicture}[scale=1]
    \begin{axis}[
    xlabel=Time,ylabel=Volt
    , grid style={draw=gray!20}, grid = both, minor tick num = 4 
    , height = \height
    ]
    \addplot [color=blue] gnuplot [ raw gnuplot ] {
        plot "./output_1hz.txt" using 2:3 with lines;
    };
    \end{axis}
\end{tikzpicture}
\begin{tikzpicture}[scale=1]
    \begin{axis}[
    xlabel=Time,ylabel=Volt
    , grid style={draw=gray!20}, grid = both, minor tick num = 4 
    , height = \height
    , xmax = 0.2
    ]
    \addplot [color=blue] gnuplot [ raw gnuplot ] {
        plot "./output_10hz.txt" using 2:3 with lines;
    };
    \end{axis}
\end{tikzpicture}

\begin{tikzpicture}[scale=1]
    \begin{axis}[
    xlabel=Time,ylabel=Volt
    , grid style={draw=gray!20}, grid = both, minor tick num = 4 
    , height = \height
    , xmax = 0.06
    ]
    \addplot [color=blue] gnuplot [ raw gnuplot ] {
        set xrange [0:0.1];
        plot "./output_100hz.txt" using 2:3 with lines;
    };
    \end{axis}
\end{tikzpicture}
\begin{tikzpicture}[scale=1]
    \begin{axis}[
    xlabel=Time,ylabel=Volt
    , grid style={draw=gray!20}, grid = both, minor tick num = 4 
    , height = \height
    , xmax = 0.006
    ]
    \addplot [color=blue] gnuplot [ raw gnuplot ] {
        plot "./output_1000hz.txt" using 2:3 with lines;
    };
    \end{axis}
\end{tikzpicture}



\paragraph{Problem 5 - 3 Points}
Draw the equivalent circuit of a myelinated axon segment.

\paragraph{Problem 6 - 7 Points}
A Purkinje neuron with 8 levels of branching obeys Rall’s Law. If the primary
dendrite is 8 microns in diameter, what is the diameter of the final branches?
I take 1 micron of dendrite each from the above Purkinje neuron primary
dendrite, and final dendrite. What is the ratio of axial resistances of these
bits of dendrite?


\end{document}          

